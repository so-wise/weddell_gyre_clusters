%%%%%%%%%%%%%%%%%%%%%%%%%%%%%%%%%%%%%%%%%
% Lachaise Assignment
% LaTeX Template
% Version 1.0 (26/6/2018)
%
% This template originates from:
% http://www.LaTeXTemplates.com
%
% Authors:
% Marion Lachaise & François Févotte
% Vel (vel@LaTeXTemplates.com)
%
% License:
% CC BY-NC-SA 3.0 (http://creativecommons.org/licenses/by-nc-sa/3.0/)
% 
%%%%%%%%%%%%%%%%%%%%%%%%%%%%%%%%%%%%%%%%%

%----------------------------------------------------------------------------------------
%	PACKAGES AND OTHER DOCUMENT CONFIGURATIONS
%----------------------------------------------------------------------------------------

\documentclass{article}

\input{structure.tex} % Include the file specifying the document structure and custom commands

%----------------------------------------------------------------------------------------
%	ASSIGNMENT INFORMATION
%----------------------------------------------------------------------------------------

\title{Unsupervised classification of Weddell Gyre profiles} % Title of the assignment

\author{D. Jones*, M. Sonnewald, I. Rosso, S. Zhou, L. Boehme\\ \texttt{*dannes@bas.ac.uk}} % Author name and email address

\date{Report \today} % University, school and/or department name(s) and a date

%----------------------------------------------------------------------------------------

\begin{document}

\maketitle % Print the title

%----------------------------------------------------------------------------------------
%	INTRODUCTION
%----------------------------------------------------------------------------------------

\section*{Introduction} % Unnumbered section

This is where the introduction will go. 

% Math equation/formula
\begin{equation}
	I = \int_{a}^{b} f(x) \; \text{d}x.
\end{equation}

Perhaps some more text here 

\begin{info} % Information block
This is a cool information block thingy
\end{info}

%----------------------------------------------------------------------------------------
%	PROBLEM 1
%----------------------------------------------------------------------------------------

\section{Section title} % Numbered section

Yep

%------------------------------------------------

\subsection{Subsection title}

% Numbered question, with subquestions in an enumerate environment
\begin{question}
Question to be addressed could go here. 

	% Subquestions numbered with letters
	\begin{enumerate}[(a)]
		\item Do this.
		\item Do that.
		\item Do something else.
	\end{enumerate}
\end{question}
	
%------------------------------------------------

\subsection{Algorithmic issues}

How cool, we have an algorithm LaTeX thingy for pseudocode

\begin{center}
	\begin{minipage}{0.5\linewidth} % Adjust the minipage width to accomodate for the length of algorithm lines
		\begin{algorithm}[H]
			\KwIn{$(a, b)$, two floating-point numbers}  % Algorithm inputs
			\KwResult{$(c, d)$, such that $a+b = c + d$} % Algorithm outputs/results
			\medskip
			\If{$\vert b\vert > \vert a\vert$}{
				exchange $a$ and $b$ \;
			}
			$c \leftarrow a + b$ \;
			$z \leftarrow c - a$ \;
			$d \leftarrow b - z$ \;
			{\bf return} $(c,d)$ \;
			\caption{\texttt{FastTwoSum}} % Algorithm name
			\label{alg:fastTwoSum}   % optional label to refer to
		\end{algorithm}
	\end{minipage}
\end{center}

Are these words? Of course they are. 

% Numbered question, with an optional title
\begin{question}[\itshape (things to be addressed)]
These boxes might be useful for unresolved questions or issues.
\end{question}

Some more text here 

%----------------------------------------------------------------------------------------
%	PROBLEM 2
%----------------------------------------------------------------------------------------

\section{Implementation}

Here is some code

% File contents
\begin{file}[hello.py]
\begin{lstlisting}[language=Python]
#! /usr/bin/python

import sys
sys.stdout.write("Hello World!\n")
\end{lstlisting}
\end{file}

Some other code

% Command-line "screenshot"
\begin{commandline}
	\begin{verbatim}
		$ chmod +x hello.py
		$ ./hello.py

		Hello World!
	\end{verbatim}
\end{commandline}

I'm not sure if this is needed, but it looks cool

% Warning text, with a custom title
\begin{warn}[Notice:]
It's raining out, take an umbrella
\end{warn}

%----------------------------------------------------------------------------------------

\end{document}
